\section{Implementation}\label{sec:03_impl}
% Explain section
This section explains the implementation based on the conceptual design introduced in \Sec{sec:02_design}. 
% MVC
As mentioned in PROBLM, the MVC architecture is used to implement this application.
% What is what
The models are introduced in \Sec{subsec:03_impl_models}, the views and controllers in \Sec{subsec:03_impl_servlets}.
% Others
Additionally, this implementation uses custom object stores (introduced in \Sec{subsec:03_impl_objstores}) and filters (introduced in \Sec{subsec:03_impl_filters}).


% !!!!
% MVC
% JAVA BEANS
% ...

% Object Store (User, Rooms, all Models + Beans)


\subsection{Models}\label{subsec:03_impl_models}
% Intro
As being mentioned, this project is implemented using the MVC pattern. Therefore, each entity of the chat system is represented by a model.
% Beans
Each model is implemented using the Java Bean specification. This enables to reuse models in \texttt{JSP} files.

% Figure
\begin{figure}[h]
\centering
\includegraphics[scale=0.8]{images/03_impl/models}
\caption{Used models for the chat system}
\label{fig:03_impl_models_models}
\end{figure}

% Explain Figure
\Fig{fig:03_impl_models_models} shows all models used in the implementation of the chat system.
% Room and message
A \texttt{Room} models exists, which represents a room where users can chat with each other. Therefore, multiple messages are saved in a \texttt{Room}. A \texttt{Message} model represents a message send by any user in a specific room. Each \texttt{Message} is identified by its message-text, the name of the user, and the date when it was send.
% User
Each user is represented by a \texttt{User} model. This model is identified by the username, and the password.


% ==============
% ==============
\subsection{Object Stores}\label{subsec:03_impl_objstores}
% Why
As being mentioned in SEC DESIGN STORAGE, a custom store is needed to save the in \Sec{subsec:03_impl_models} introduced models.

Therefore, the following stores are being implemented:
\begin{itemize}
\item \texttt{UserStore}, saves \texttt{User} models
\item \texttt{RoomStore}, saves \texttt{Room} models
\end{itemize}
% Store functionality
Because a user and room can only exists once, a \texttt{Set}\footnote{Set (Java Platform SE 8) - \url{https://docs.oracle.com/javase/8/docs/api/java/util/Set.html}} data structure is used to save both Users, and Rooms. Furthermore, a store needs to provide the functionality to add an object to the store, get all objects from the store, or a specific one, and to check if an object has already been added to the store.


% Abstract class
Therefore, an abstract class called \texttt{ObjectsStore} is implemented, which implements the previously mentioned functionalities.
% Generics
The \texttt{ObjectStore} uses generics to implement the properties and methods. The \texttt{UserStore} and \texttt{RoomStore} implement the \texttt{ObjectStore} class and set the model (\texttt{User} and \texttt{Room}, respectively) for the generic type.
% Lifetime
The lifetime of an object saved to a store, is equal to the lifetime of the running webserver. As long as the webserver is running, the objects are saved to memory. If the webserver gets shout down, all rooms and messages are being removed from the memory.
% Lifetime of user
Except for the \texttt{UserStore}, which reads all \texttt{User} models from a text file, and writes newly created Users to a text file. Therefore, when the server restarts, all previously added \texttt{User} models are available again.

% Singleton pattern
To make the same set of Users, and Rooms available for each single servlet instance, the singleton pattern is used for the stores. This ensures, that all servlets work with the same set of Users and Rooms. An example of the singleton usage is shown in \Lst{lst:03_impl_objstores_singleton}.
% Singleton example
\begin{lstlisting}[label=lst:03_impl_objstores_singleton, caption=Example usage of the singleton pattern, language=java]
Set<Room> rooms = RoomStore.getInstance().getAll();
\end{lstlisting}


% ==============
% ==============
\subsection{Servlets}\label{subsec:03_impl_servlets}
Fig XY shows the architecture of the in SEC DESIGN introduced routes. To implement this architecture, the following Servlets are implemented:
\begin{itemize}
\item \texttt{AuthLoginServlet}
\item \texttt{AuthLogoutServlet}
\item \texttt{LoginServlet}
\item \texttt{UserPageServlet}
\item \texttt{RoomServlet}
\item \texttt{RoomCreateServlet}
\item \texttt{AdminServlet}
\end{itemize}
% Views
Most of the mentioned servlets, own a specific view (a JSP file) which is saved in \path{Chat/src/main/webapp/views}.


\subsubsection{Authentication}\label{subsubsec:03_impl_servlets_auth}
For authentication, the AuthLoginServlet and the AuthLogoutServlet are created.
% Login
The \texttt{AuthLoginServlet} requires a \texttt{POST} request, and validates the request accordingly. If the given username and password are valid credentials (the user with the given name exist, and the password is correct), the \texttt{AuthLoginServlet} creates a new \texttt{HTTPSession}, sets the active user, and the attribute \texttt{is\_authenticated} to \texttt{true}. This property is important for the \texttt{AuthFilter} introduced in \Sec{subsubsec:03_impl_filters_auth}. After that, the \texttt{AuthLoginServlet} redirects the user to the \textit{User-Page}.
% Invalid credentials
If the \texttt{POST} request is invalid (no username and password set), the \texttt{AuthLoginServlet} sends a \texttt{400 - Bad Request} HTTP error. Otherwise, if the request is valid, but the credentials are invalid (wrong username or wrong password), the \texttt{AuthLoginServlet} redirects back to the \textit{Login-Page}.

% Logout
To logout, the \texttt{AuthLogoutServlet} exist. If a \texttt{GET} request is made from an active user, it invalidates the \texttt{HTTPSession} and redirects to the \textit{Login-Page}.


\subsubsection{Login}\label{subsubsec:03_impl_servlets_login}
% Figure
\begin{figure}[h]
\centering
\includegraphics[scale=0.5]{images/03_impl/login/login_form}
\caption{The login HTML form}
\label{fig:03_impl_servlets_login_form}
\end{figure}
% Explain figure
\Fig{fig:03_impl_servlets_login_form} shows the \textit{Login-Page}. It consists of a single HTML form which sends a \texttt{POST} request to the \texttt{AuthLoginServlet} (introduced before in \Sec{subsubsec:03_impl_servlets_auth}) consisting of the username and the corresponding password.


\subsubsection{User}\label{subsubsec:03_impl_servlets_user}
% Figure
\begin{figure}[h]
\centering
\includegraphics[scale=0.2]{images/03_impl/user/userpage_before_after}
\caption{Creation-process of \textit{user3}}
\label{fig:03_impl_servlets_user_beforeafter}
\end{figure}
% Explain figure
\Fig{fig:03_impl_servlets_user_beforeafter} shows the \textit{User-Page}. It is shown that the \textit{User-Page} lists all available rooms, otherwise a message if no rooms exists.
% How
The view reads all available rooms from the \texttt{RoomStore} (introduced in SEC ROOMSTORE) and uses a \texttt{for} loop to list the rooms as clickable links in a \texttt{ul} element, which is shown in \Lst{lst:03_impl_servlets_user}.
% List rooms
\begin{lstlisting}[label=lst:03_impl_servlets_user, caption=List all available rooms, language=html]
<%@ page import="java.util.Set" %>
<%@ page import="it.unitn.disi.webarch.chat.helper.RoomStore" %>
<%@ page import="it.unitn.disi.webarch.chat.models.room.Room" %>
<%
    Set<Room> rooms = RoomStore.getInstance().getAll();
%>
...
<body>
<ul>
  <% for(Room room: rooms) { %>
    <li><a href="<% request.getContextPath(); %>/room/<%= room.getName() %>"><%= room.getName() %></a></li>
  <% } %>
</ul>
</body>
\end{lstlisting}


\subsubsection{Room}\label{subsubsec:03_impl_servlets_room}
% Figure
\begin{figure}[h]
\centering
\includegraphics[scale=0.3]{images/03_impl/room/chat_all_steps}
\caption{Creation-process of \textit{user3}}
\label{fig:03_impl_servlets_admin_chat}
\end{figure}
% Explain figure
In a \textit{Room-Page}, multiple users can chat with each other, which is shown in \Fig{fig:03_impl_servlets_admin_chat}.
% The view
The \texttt{RoomServlet} owns a view called \texttt{Room.jsp}, which shows an HTML-Form, and lists all messages of the active Room.

% The active room
The active room is set as a Java Bean in the request. Therefore, the \texttt{JSP} view can access the model of the active room via \texttt{jsp:getProperty}. Then, all messages which belongs to the active room can be received via the \texttt{getMessages()} method of a \texttt{Room} model and can be listed in a \texttt{ul} element using a \texttt{for} loop. \Lst{lst:03_impl_servlets_room_bean} shows the implementation, how messages are listed in the room view.
% The listing
\begin{lstlisting}[label=lst:03_impl_servlets_room_bean, caption=List messages for a specific room, language=html]
<%@ page import="java.util.List" %>
<%@ page import="it.unitn.disi.webarch.chat.models.room.Message" %>
<jsp:useBean id="activeRoom" class="it.unitn.disi.webarch.chat.models.room.Room" scope="request" />
...
<body>
<%
List<Message> messages = activeRoom.getMessages();
for(Message message: messages) {
%>
  <div>
    <p><em><%= message.getUser() %> at <%= message.getFormattedDate() %>:</em></p>
    <p><%= message.getMessage() %></p>
  </div>
<% } %>
</body>
\end{lstlisting}

% Send messages
The HTML-form sends a \texttt{POST} request to itself. The request consists of an attribute called \texttt{message}, which is the message text of the user. After a \texttt{POST} request has been made, the \texttt{RoomServlet} constructs a new \texttt{Message} model, using the message text received from the \texttt{POST} request, and the username of the active user which is saved in the HTTP session. After that, the newly created \texttt{Message} model is added to the active \texttt{Room} model via the \texttt{addMessage} method. Finally, the page gets reloaded using the \texttt{doGet} method of the servlet. If the request is not valid (no room requested), the \texttt{RoomServlet} responses with a \texttt{400 - Bad Request} HTTP error. Otherwise, if the requested room does not exists, the \texttt{RoomServlet} responses a \texttt{404 - Not Found} HTTP error.

% Reload
To update newly created messages, the view reloads itself every 15 seconds using \texttt{<meta http-equiv="refresh" content="15">}.


\newpage
\subsubsection{Admin}\label{subsubsec:03_impl_servlets_admin}
% Figure
\begin{figure}[h]
\centering
\includegraphics[scale=0.2]{images/03_impl/admin/create_user_all}
\caption{Creation-process of \textit{user3}}
\label{fig:03_impl_servlets_room_createall}
\end{figure}
% Explain figure
\Fig{fig:03_impl_servlets_room_createall} illustrates the successful creation-process of a user.
% HTML for
The admin can fill out the HTML-form of the \textit{Admin-Page}. After clicking submit, it will send the data (\texttt{username} and \texttt{password}) via a \texttt{POST} request to itself.
% Servlet
The \texttt{AdminServlet}, validates the the \texttt{POST} request attributes and checks if the user already exist If the user does not exist, the user will be add via the \texttt{addUser} method of the \texttt{UserStore}. As mentioned in SEC USERSTORE, the \texttt{UserStore} will write the user to the \texttt{users.txt} file.
% Check credentials
To check if the given credentials are valid, the \texttt{AdminServlet} checks if the given \texttt{username} and \texttt{password} are not \texttt{null}. Otherwise, the \texttt{AdminServlet} will response with a \texttt{400 - Bad Request} HTTP error. Additionally, the \texttt{AdminServlet} checks if the length of the \texttt{username} and \texttt{password} is bigger than 0, and if the \texttt{username} does not equal \textit{admin}. If one condition is not true, the \texttt{AdminServlet} executes the \texttt{doGet} method to reload itself.



% ==============
% ==============
\subsection{Filters}\label{subsec:03_impl_filters}

\subsubsection{Authentication}\label{subsubsec:03_impl_filters_auth}

\subsubsection{Admin}\label{subsubsec:03_impl_filters_admin}
