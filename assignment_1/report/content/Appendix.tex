\section{Part 1}

\subsection{Implementation}

\subsubsection{StringReverser}

\begin{lstlisting}[label=lst:01_part1_impl_stringreverser_code, caption=CLI command to start a GitLab runner in a Docker container, language=java, numbers=none]
package it.unitn.disi.webarch.assignment1;

public class StringReverser {

    public static void main(String[] args) {
        if (args.length >= 1) {
            String text = args[0];

            if (text != null) {
                String reversedString = reverseString(text);
                System.out.println(reversedString);
            }
        } else {
            System.out.println("A string to reverse is required.");
            System.exit(0);
        }
    }

    /**
     * This method reverses the given string.
     *
     * @param text
     * @return Reversed text
     */
    private static String reverseString(String text) {
        StringBuilder stringBuilder = new StringBuilder(text);
        stringBuilder.reverse();
        return stringBuilder.toString();
    }

}
\end{lstlisting}

\subsubsection{TinyHttpd}

\begin{lstlisting}[label=lst:01_part1_impl_tinyhttpd_requestparser, caption=CLI command to start a GitLab runner in a Docker container, language=java, numbers=none]
package it.unitn.disi.webarch.tinyhttpd.utils;

import java.util.StringTokenizer;

public class RequestParser {

    private String request;
    private String method;
    private String protocol;
    private String fullPath;
    private String path;
    private String queryString;

    /**
     * The RequestParser is able to parse an HTTP request
     * into its single parts.
     *
     * @param request
     */
    public RequestParser(String request) throws Exception {
        this.request = request;
        this.parseRequest();
        this.parsePath();
    }

    private void parseRequest() throws Exception {
        StringTokenizer tokenizer = new StringTokenizer(this.request);
        if (tokenizer.countTokens() < 3) {
            throw new Exception("The request \"" + this.request + "\" is invalid");
        }

        this.method = tokenizer.nextToken();
        this.fullPath = tokenizer.nextToken();
        this.protocol = tokenizer.nextToken();
    }

    private void parsePath() {
        StringTokenizer tokenizer = new StringTokenizer(this.fullPath, "?");

        this.path = tokenizer.nextToken();
        if (tokenizer.hasMoreTokens()) {
            this.queryString = tokenizer.nextToken();
        }
    }

    public String getMethod() {
        return this.method;
    }

    public String getProtocol() {
        return this.protocol;
    }

    public String getPath() {
        return this.path;
    }

    public String getQueryString() {
        return this.queryString;
    }
}
\end{lstlisting}

\begin{lstlisting}[label=lst:01_part1_impl_tinyhttpd_commandfactory, caption=CLI command to start a GitLab runner in a Docker container, language=java, numbers=none]
package it.unitn.disi.webarch.tinyhttpd.utils;

import java.util.StringTokenizer;

public class CommandFactory {

    /**
     * Generates the command for the given process name.
     *
     * @param process Name of the requested process
     * @param parameters Probably the query string
     * @return Command as string
     * @throws Exception If the process name is unknown
     */
    public static String generateCommand(String process, String parameters) throws Exception {
        if (process.equals("reverse")) {
            return generateReverseProcessCommand(parameters);
        } else {
            throw new Exception("No process called \"" + process + "\" available");
        }
    }

    private static String generateReverseProcessCommand(String parameters) throws Exception {
        StringTokenizer paramTokenizer = new StringTokenizer(parameters, "&");
        // first check if there are any parameters
        if (!paramTokenizer.hasMoreTokens()) {
            throw new Exception("No parameters given. The reverse process needs at least one parameter");
        }
        String firstParam = paramTokenizer.nextToken();
        String[] paramKeyValue = firstParam.split("=");
        if (paramKeyValue.length >= 2) {
            // We need at least 2 elements
            String textToReverse = paramKeyValue[1];
            String artifactPath = System.getProperty("user.dir") + "/jars/StringReverser.jar";
            String command = "java" + " -jar " + artifactPath + " \"" + textToReverse + "\"";

            return command;
        } else {
            throw new Exception("No valid parameter input given for parameters \"" + parameters + "\"");
        }
    }

}
\end{lstlisting}

\begin{lstlisting}[label=lst:01_part1_impl_tinyhttpd_launchprocess, caption=CLI command to start a GitLab runner in a Docker container, language=java, numbers=none]
private String launchProcess(String command) throws Exception {
    ProcessBuilder processBuilder = new ProcessBuilder();
    processBuilder.command("bash", "-c", command);
    Process process = processBuilder.start();

    StringBuilder output = new StringBuilder();
    BufferedReader reader = new BufferedReader(
            new InputStreamReader(process.getInputStream()));

    String line;
    while ((line = reader.readLine()) != null) {
        output.append(line + "\n");
    }

    return output.toString();
}
\end{lstlisting}

\begin{lstlisting}[label=lst:01_part1_impl_tinyhttpd_sendSuccessResponseHeader, caption=CLI command to start a GitLab runner in a Docker container, language=java, numbers=none]
private void sendSuccessResponseHeader(int responseLength, String mimeType) {
    ps.print("HTTP/1.1 200 OK\r\n");
    ps.print("Content-Length: " + responseLength + "\r\n");
    ps.print("Content-Type: " + mimeType + "\r\n");
    ps.print("\r\n");
}
\end{lstlisting}

\section{Part 2}

\subsection{Implementation}

\begin{lstlisting}[label=lst:02_part2_impl_script, caption=CLI command to start a GitLab runner in a Docker container, language=bash, numbers=none]
#!/bin/sh

send_response () {
  echo "Content-type: text/plain; charset=iso-8859-1"
  echo "Content-Length: ${#1}"
  echo
  echo $1
}


# Check if the request method is GET
if [ $REQUEST_METHOD == "GET" ]; then
  # Check if query is given
  if [ ! $QUERY_STRING ]; then
    send_response "No query given"
  else
    # Parse query
    saveIFS=$IFS
    IFS='=&'
    params=($QUERY_STRING)
    IFS=$saveIFS
    STRING_TO_REVERSE=${params[1]}

    # Check if the given parameter is valid
    if [ ! $STRING_TO_REVERSE ]; then
      send_response "No valid string to reverse is given"
    else
      REVERSE_JAVA_ARTIFACT="$(pwd)/StringReverser.jar"
      REVERSED_STRING=$(java -jar $REVERSE_JAVA_ARTIFACT $STRING_TO_REVERSE)

      send_response $REVERSED_STRING
    fi
  fi
else
  send_response "Only GET requests are allowed"
fi
\end{lstlisting}
