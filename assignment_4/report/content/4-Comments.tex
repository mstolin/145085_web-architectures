\section{Comments and Notes}\label{sec:04_comments}
% Intro
This section describes the problems encountered during the development of the application.


\subsection{Routing}\label{subsec:04_comments_routing}
% Whats the problem
A requirement was to deploy the \textit{Angular} application on a \textit{Tomcat} webserver. However, \textit{Angular} redirects all requests to the \path{index.html}.
% Example
For example, requesting the page \url{http://localhost:8080/parliament/detail/1735} will not work, because a directory \path{detail/1735} does not exists. Then, the \textit{Tomcat} webserver responses a 404 error page.

% Solution
The solution to this problem, is to use the \texttt{HashLocationStrategy}\footnote{\url{https://angular.io/api/common/HashLocationStrategy}}. \Lst{lst:04_comments_routing_hashrouting} shows how it is implemented in the application.
% What happens
Then, the previously mentioned URL can be accessed as \url{http://localhost:8080/parliament/#/detail/1735}.
% Implementation
\begin{lstlisting}[label=lst:04_comments_routing_hashrouting, caption=Application routing configuration of \texttt{app-routing.module.ts}, language=java]
const routes: Routes = [
  { path: '', component: MemberListComponent },
  { path: 'detail/:memberId', component: MemberDetailComponent }
];

@NgModule({
  imports: [RouterModule.forRoot(routes, {useHash: true})],
  exports: [RouterModule]
})
\end{lstlisting}


\subsection{Building the Angular Application}\label{subsec:04_comments_building}
% What
Another problem is how to build the \textit{Angular} application, which is also related to the \textit{Tomcat} webserver.
The \textit{Tomcat} webserver requires static files (e.g.: HTML files, Javascript libraries) to be located at \path{PROJECT\_ROOT/src/main/webapp}.
Therefore, whenever the angular application is built, the output has to be saved in this directory.

% Solution
To solve this issue, \textit{Angular} allows to define the \texttt{outputPath} for the application when using \texttt{ng build}. \Lst{lst:04_comments_building_config} shows the \path{angular.json} configuration.
% Config
\begin{lstlisting}[label=lst:04_comments_building_config, caption=\texttt{angular.json} configuration]
...
"build": {
  "builder": "@angular-devkit/build-angular:browser",
  "options": {
    "outputPath": "../../webapp",
...
\end{lstlisting}


% Base href
Additionally, it is important to define the \textit{base-href}, that defines the the path where the application is located on the webserver. This application is supposed to be available via the path \path{/parliament} (e.g.: \url{http://localhost:8080/parliament/}).


Therefore, the command \texttt{\$ ng build -{}-base-href /parliament/} has to be used to build the application.
